\usepackage[utf8]{inputenc}
\usepackage[T1]{fontenc}
\usepackage{setspace}
\doublespacing
\usepackage{fullpage}
\usepackage[pdftex]{graphicx}
\graphicspath{{figures/}}
\usepackage{xcolor}

% Verbatim with soft grey background.
% Taken from Enrico Gregorio and Lightness Races in Orbit at
% https://tex.stackexchange.com/questions/141100/getting-verbatim-with-soft-grey-background-as-in-tex-stackexchange/141128#141128
\usepackage{newverbs}
\definecolor{cverbbg}{gray}{0.93}
\newverbcommand{\cverb}
  {\setbox\verbbox\hbox\bgroup}
  {\egroup\colorbox{cverbbg}{\box\verbbox}}
\newenvironment{lcverbatim}
 {\SaveVerbatim{cverb}}
 {\endSaveVerbatim
  \flushleft\fboxrule=0pt\fboxsep=.5em
  \colorbox{cverbbg}{%
    \makebox[\dimexpr\linewidth-2\fboxsep][l]{\BUseVerbatim{cverb}}%
  }
  \endflushleft
}

\usepackage{booktabs}
\usepackage[numbers]{natbib}
\usepackage{doi}
\usepackage{hyperref}
\definecolor{darkblue}{rgb}{.15,0,.7}
\hypersetup{
  colorlinks=true,
  linkcolor=darkblue,
  citecolor=darkblue,
  urlcolor=darkblue
}
\usepackage{grffile}% To solve the "unknown graphic extension" issue.
\usepackage{etoolbox}
\usepackage{amsmath}
\usepackage{amssymb}
\DeclareMathOperator*{\minimize}{minimize}%
\DeclareMathOperator*{\sgn}{sgn}%
\DeclareMathOperator*{\supp}{supp}%
\DeclareMathOperator*{\argmin}{arg\,min}%
\DeclareMathOperator*{\hessian}{H}%
\DeclareMathOperator*{\var}{Var}
\newcommand{\defeq}{\triangleq}
\newcommand{\reals}{\mathbb{R}}% Set of real numbers.
\newcommand{\naturalnumbers}{\mathbb N}% Natural numbers.
\renewcommand{\Pr}{\operatorname{\mathbb P}}% Probability.
\newcommand{\E}{\operatorname{\mathbb E}}% Mathematical expectation.
\newcommand{\law}{\operatorname{\mathcal L}}% Law.
\newcommand{\fedeq}{=:}
\newcommand{\ind}[2]{\left(#1\colon #2\right)} % Indexed family.
\usepackage{amsthm}
\newtheorem{exam}{Example}
\newtheorem{definition}{Definition}
\newtheorem{remark}{Remark}
\newtheorem{prop}{Proposition}
\newtheorem{cor}{Corollary}
\newtheorem{lemma}{Lemma}
\newtheorem{conj}{Conjecture}
\newcommand{\indicator}[1]{\mathbb{I}(#1)}% Indicator function.
\newcommand{\set}[2]{\left\{#1\colon #2\right\}}% Set builder.
\renewcommand{\d}{\mathop{}\!\mathrm{d}}% by egrep at http://tex.stackexchange.com/questions/5511/good-practice-on-spacing
\renewcommand{\hat}{\widehat}
\newcommand{\setvar}[1]{\left\{#1\right\}}% Enumerative set.
\usepackage{subcaption}
